% Tikz library for DFT elements
% -----------------------------
% Authors:
% - Dennis Guck (initial version)
% - Sebastian Junges (extensions)
%

\makeatletter

\pgfdeclareshape{spare}{
  % The 'minimum width' and 'minimum height' keys, not the content, determine
  % the size
  \savedanchor\northeast{%
    \pgfmathsetlength\pgf@x{\pgfshapeminwidth}%
    \pgfmathsetlength\pgf@y{\pgfshapeminheight}%
    \pgf@x=0.5\pgf@x
    \pgf@y=0.5\pgf@y
  }
  % This is redundant, but makes some things easier:
  \savedanchor\southwest{%
    \pgfmathsetlength\pgf@x{\pgfshapeminwidth}%
    \pgfmathsetlength\pgf@y{\pgfshapeminheight}%
    \pgf@x=-0.5\pgf@x
    \pgf@y=-0.5\pgf@y
  }
  % Inherit from rectangle
  \inheritanchorborder[from=rectangle]

  % Define same anchor a normal rectangle has
  \anchor{center}{\pgfpointorigin}
  \anchor{north}{\northeast \pgf@x=0pt}
  \anchor{east}{\northeast \pgf@y=0pt}
  \anchor{south}{\southwest \pgf@x=0pt}
  \anchor{west}{\southwest \pgf@y=0pt}
  \anchor{north east}{\northeast}
  \anchor{north west}{\northeast \pgf@x=-\pgf@x}
  \anchor{south west}{\southwest}
  \anchor{south east}{\southwest \pgf@x=-\pgf@x}
  \anchor{text}{
    \pgfpointorigin
    \advance\pgf@x by -.5\wd\pgfnodeparttextbox%
    \advance\pgf@y by -.5\ht\pgfnodeparttextbox%
    \advance\pgf@y by +.5\dp\pgfnodeparttextbox%
  }

  % Define anchors for signal ports
  \anchor{O}{ % output anchor
    \pgf@process{\northeast}%
    \pgf@x=0pt%
  }
  \anchor{P}{ % primary anchor
    \pgf@process{\northeast}%
    \pgf@x=-.5\pgf@x%
    \pgf@y=-\pgf@y%
  }
  \anchor{SA}{ % spare anchor
    \pgf@process{\northeast}%
    \pgf@x=.1\pgf@x%
    \pgf@y=-\pgf@y%
  }
  \anchor{SB}{ % spare anchor
    \pgf@process{\northeast}%
    \pgf@x=.3\pgf@x%
    \pgf@y=-\pgf@y%
  }
  \anchor{SC}{ % spare anchor
    \pgf@process{\northeast}%
    \pgf@x=.5\pgf@x%
    \pgf@y=-\pgf@y%
  }
  \anchor{SD}{ % spare anchor
    \pgf@process{\northeast}%
    \pgf@x=.7\pgf@x%
    \pgf@y=-\pgf@y%
  }
  \anchor{SE}{ % spare anchor
    \pgf@process{\northeast}%
    \pgf@x=.9\pgf@x%
    \pgf@y=-\pgf@y%
  }
  \anchor{BOX}{ % spare box anchor
    \pgf@process{\northeast}%
    \pgf@x=\pgf@x%
    \pgf@y=-\pgf@y%
  }
  \anchor{LINE}{
    \pgf@process{\northeast}%
    \pgf@x=-1\pgf@x%
    \pgf@y=.\pgf@y%
  }
  % Draw the rectangle box and the port labels
  \backgroundpath{
    % Rectangle box
    \pgfpathrectanglecorners{\southwest}{\northeast}
    % Line in the middle
    \pgf@anchor@spare@LINE
    \pgf@xa=\pgf@x \pgf@ya=\pgf@y
    \pgf@xb=\pgf@x \pgf@yb=\pgf@y
    \advance\pgf@xb by -2\pgf@x
    \pgfpathmoveto{\pgfpoint{\pgf@xa}{\pgf@ya}}
    \pgfpathlineto{\pgfpoint{\pgf@xb}{\pgf@ya}}
    % Box for spare parts
    \pgf@anchor@spare@BOX
    \pgf@xa=\pgf@x \pgf@ya=\pgf@y
    \pgf@xb=\pgf@x \pgf@yb=\pgf@y
    \pgf@xc=\pgf@x \pgf@yc=\pgf@y
    \advance\pgf@ya by -.\pgf@y
    \advance\pgf@yb by -.8\pgf@y
    \advance\pgf@xb by -1\pgf@x
    \pgfpathmoveto{\pgfpoint{\pgf@xa}{\pgf@ya}}
    \pgfpathlineto{\pgfpoint{\pgf@xa}{\pgf@yb}}
    \pgfpathlineto{\pgfpoint{\pgf@xb}{\pgf@yb}}
    \pgfpathlineto{\pgfpoint{\pgf@xb}{\pgf@ya}}
    \pgfclosepath

    % Draw port labels
    \begingroup
    \tikzset{spare/port labels} % Use font from this style
    \tikz@textfont

    \pgf@anchor@spare@O
    \pgftext[top,at={\pgfpoint{\pgf@x}{\pgf@y}},y=-\pgfshapeinnerysep]{}

    \pgf@anchor@spare@P 
    \pgftext[bottom,at={\pgfpoint{\pgf@x}{\pgf@y}},y=\pgfshapeinnerysep]{}
    
    \pgf@anchor@spare@SA
    \pgftext[bottom,at={\pgfpoint{\pgf@x}{\pgf@y}},y=\pgfshapeinnerysep]{}
    
    \pgf@anchor@spare@SB
    \pgftext[bottom,at={\pgfpoint{\pgf@x}{\pgf@y}},y=\pgfshapeinnerysep]{}
    
    \pgf@anchor@spare@SC
    \pgftext[bottom,at={\pgfpoint{\pgf@x}{\pgf@y}},y=\pgfshapeinnerysep]{}
    
    \pgf@anchor@spare@SD
    \pgftext[bottom,at={\pgfpoint{\pgf@x}{\pgf@y}},y=\pgfshapeinnerysep]{}
    
    \pgf@anchor@spare@SE
    \pgftext[bottom,at={\pgfpoint{\pgf@x}{\pgf@y}},y=\pgfshapeinnerysep]{}
    \endgroup
  }
}

\tikzset{add font/.code={\expandafter\def\expandafter\tikz@textfont\expandafter{\tikz@textfont#1}}} 

\tikzset{spare/port labels/.style={font=\sffamily\scriptsize}}
\tikzset{every spare node/.style={draw,minimum height=1cm,minimum 
width=1.5cm,fill=white!100!black,inner sep=1mm,outer sep=0pt,cap=round,add 
font=\sffamily}}

\pgfdeclareshape{spareM}{
  % The 'minimum width' and 'minimum height' keys, not the content, determine
  % the size
  \savedanchor\northeast{%
    \pgfmathsetlength\pgf@x{\pgfshapeminwidth}%
    \pgfmathsetlength\pgf@y{\pgfshapeminheight}%
    \pgf@x=0.5\pgf@x
    \pgf@y=0.5\pgf@y
  }
  % This is redundant, but makes some things easier:
  \savedanchor\southwest{%
    \pgfmathsetlength\pgf@x{\pgfshapeminwidth}%
    \pgfmathsetlength\pgf@y{\pgfshapeminheight}%
    \pgf@x=-0.5\pgf@x
    \pgf@y=-0.5\pgf@y
  }
  % Inherit from rectangle
  \inheritanchorborder[from=rectangle]

  % Define same anchor a normal rectangle has
  \anchor{center}{\pgfpointorigin}
  \anchor{north}{\northeast \pgf@x=0pt}
  \anchor{east}{\northeast \pgf@y=0pt}
  \anchor{south}{\southwest \pgf@x=0pt}
  \anchor{west}{\southwest \pgf@y=0pt}
  \anchor{north east}{\northeast}
  \anchor{north west}{\northeast \pgf@x=-\pgf@x}
  \anchor{south west}{\southwest}
  \anchor{south east}{\southwest \pgf@x=-\pgf@x}
  \anchor{text}{
    \pgfpointorigin
    \advance\pgf@x by -.5\wd\pgfnodeparttextbox%
    \advance\pgf@y by -.5\ht\pgfnodeparttextbox%
    \advance\pgf@y by +.5\dp\pgfnodeparttextbox%
  }

  % Define anchors for signal ports
  \anchor{O}{ % output anchor
    \pgf@process{\northeast}%
    \pgf@x=0pt%
  }
  \anchor{P}{ % primary anchor
    \pgf@process{\northeast}%
    \pgf@x=.5\pgf@x%
    \pgf@y=-\pgf@y%
  }
  \anchor{SA}{ % spare anchor
    \pgf@process{\northeast}%
    \pgf@x=-.1\pgf@x%
    \pgf@y=-\pgf@y%
  }
  \anchor{SB}{ % spare anchor
    \pgf@process{\northeast}%
    \pgf@x=-.3\pgf@x%
    \pgf@y=-\pgf@y%
  }
  \anchor{SC}{ % spare anchor
    \pgf@process{\northeast}%
    \pgf@x=-.5\pgf@x%
    \pgf@y=-\pgf@y%
  }
  \anchor{SD}{ % spare anchor
    \pgf@process{\northeast}%
    \pgf@x=-.7\pgf@x%
    \pgf@y=-\pgf@y%
  }
  \anchor{SE}{ % spare anchor
    \pgf@process{\northeast}%
    \pgf@x=.9\pgf@x%
    \pgf@y=-\pgf@y%
  }
  \anchor{BOX}{ % spare box anchor
    \pgf@process{\northwest}%
    \pgf@x=\pgf@x%
    \pgf@y=-\pgf@y%
  }
  \anchor{LINE}{
    \pgf@process{\northeast}%
    \pgf@x=-1\pgf@x%
    \pgf@y=.\pgf@y%
  }
  % Draw the rectangle box and the port labels
  \backgroundpath{
    % Rectangle box
    \pgfpathrectanglecorners{\southwest}{\northeast}
    % Line in the middle
    \pgf@anchor@spare@LINE
    \pgf@xa=\pgf@x \pgf@ya=\pgf@y
    \pgf@xb=\pgf@x \pgf@yb=\pgf@y
    \advance\pgf@xb by -2\pgf@x
    \pgfpathmoveto{\pgfpoint{\pgf@xa}{\pgf@ya}}
    \pgfpathlineto{\pgfpoint{\pgf@xb}{\pgf@ya}}
    % Box for spare parts
    \pgf@anchor@spare@BOX
    \pgf@xa=\pgf@x \pgf@ya=\pgf@y
    \pgf@xb=\pgf@x \pgf@yb=\pgf@y
    \pgf@xc=\pgf@x \pgf@yc=\pgf@y
    \advance\pgf@ya by -.\pgf@y
    \advance\pgf@xa by -1\pgf@x
    \advance\pgf@yb by -.8\pgf@y
    \advance\pgf@xb by -2\pgf@x
    \pgfpathmoveto{\pgfpoint{\pgf@xa}{\pgf@ya}}
    \pgfpathlineto{\pgfpoint{\pgf@xa}{\pgf@yb}}
    \pgfpathlineto{\pgfpoint{\pgf@xb}{\pgf@yb}}
    \pgfpathlineto{\pgfpoint{\pgf@xb}{\pgf@ya}}
    \pgfclosepath

    % Draw port labels
    \begingroup
    \tikzset{spareM/port labels} % Use font from this style
    \tikz@textfont

    \pgf@anchor@spare@O
    \pgftext[top,at={\pgfpoint{\pgf@x}{\pgf@y}},y=-\pgfshapeinnerysep]{}

    \pgf@anchor@spare@P 
    \pgftext[bottom,at={\pgfpoint{\pgf@x}{\pgf@y}},y=\pgfshapeinnerysep]{}
    
    \pgf@anchor@spare@SA
    \pgftext[bottom,at={\pgfpoint{\pgf@x}{\pgf@y}},y=\pgfshapeinnerysep]{}
    
    \pgf@anchor@spare@SB
    \pgftext[bottom,at={\pgfpoint{\pgf@x}{\pgf@y}},y=\pgfshapeinnerysep]{}
    
    \pgf@anchor@spare@SC
    \pgftext[bottom,at={\pgfpoint{\pgf@x}{\pgf@y}},y=\pgfshapeinnerysep]{}
    
    \pgf@anchor@spare@SD
    \pgftext[bottom,at={\pgfpoint{\pgf@x}{\pgf@y}},y=\pgfshapeinnerysep]{}
    
    \pgf@anchor@spare@SE
    \pgftext[bottom,at={\pgfpoint{\pgf@x}{\pgf@y}},y=\pgfshapeinnerysep]{}
    \endgroup
  }
}

\tikzset{add font/.code={\expandafter\def\expandafter\tikz@textfont\expandafter{\tikz@textfont#1}}} 

\tikzset{spareM/port labels/.style={font=\sffamily\scriptsize}}
\tikzset{every spareM node/.style={draw,minimum height=1cm,minimum 
width=1.5cm,inner sep=1mm,outer sep=0pt,cap=round,add 
font=\sffamily}}

\pgfdeclareshape{fdep}{
  % The 'minimum width' and 'minimum height' keys, not the content, determine
  % the size
  \savedanchor\northeast{%
    \pgfmathsetlength\pgf@x{\pgfshapeminwidth}%
    \pgfmathsetlength\pgf@y{\pgfshapeminheight}%
    \pgf@x=0.5\pgf@x
    \pgf@y=0.5\pgf@y
  }
  % This is redundant, but makes some things easier:
  \savedanchor\southwest{%
    \pgfmathsetlength\pgf@x{\pgfshapeminwidth}%
    \pgfmathsetlength\pgf@y{\pgfshapeminheight}%
    \pgf@x=-0.5\pgf@x
    \pgf@y=-0.5\pgf@y
  }
  % Inherit from rectangle
  \inheritanchorborder[from=rectangle]

  % Define same anchor a normal rectangle has
  \anchor{center}{\pgfpointorigin}
  \anchor{north}{\northeast \pgf@x=0pt}
  \anchor{east}{\northeast \pgf@y=0pt}
  \anchor{south}{\southwest \pgf@x=0pt}
  \anchor{west}{\southwest \pgf@y=0pt}
  \anchor{north east}{\northeast}
  \anchor{north west}{\northeast \pgf@x=-\pgf@x}
  \anchor{south west}{\southwest}
  \anchor{south east}{\southwest \pgf@x=-\pgf@x}
  \anchor{text}{
    \pgfpointorigin
    \advance\pgf@x by -.5\wd\pgfnodeparttextbox%
    \advance\pgf@y by -.5\ht\pgfnodeparttextbox%
    \advance\pgf@y by +.5\dp\pgfnodeparttextbox%
  }

  % Define anchors for signal ports
  \anchor{O}{ % output anchor
    \pgf@process{\northeast}%
    \pgf@x=0pt%
  }
  \anchor{EA}{ % spare anchor
    \pgf@process{\northeast}%
    \pgf@x=.1\pgf@x%
    \pgf@y=-\pgf@y%
  }
  \anchor{EB}{ % spare anchor
    \pgf@process{\northeast}%
    \pgf@x=.3\pgf@x%
    \pgf@y=-\pgf@y%
  }
  \anchor{EC}{ % spare anchor
    \pgf@process{\northeast}%
    \pgf@x=.5\pgf@x%
    \pgf@y=-\pgf@y%
  }
  \anchor{ED}{ % spare anchor
    \pgf@process{\northeast}%
    \pgf@x=.7\pgf@x%
    \pgf@y=-\pgf@y%
  }
  \anchor{EE}{ % spare anchor
    \pgf@process{\northeast}%
    \pgf@x=.9\pgf@x%
    \pgf@y=-\pgf@y%
  }
  \anchor{T}{ % trigger anchor
    \pgf@process{\northeast}%
    \pgf@x=-1\pgf@x%
    \pgf@y=-.5\pgf@y%
  }
  \anchor{LINE}{
    \pgf@process{\northeast}%
    \pgf@x=-1\pgf@x%
    \pgf@y=.\pgf@y%
  }
  % Draw the rectangle box and the port labels
  \backgroundpath{
    % Rectangle box
    \pgfpathrectanglecorners{\southwest}{\northeast}
    % Line in the middle
    \pgf@anchor@fdep@LINE
    \pgf@xa=\pgf@x \pgf@ya=\pgf@y
    \pgf@xb=\pgf@x \pgf@yb=\pgf@y
    \advance\pgf@xb by -2\pgf@x
    \pgfpathmoveto{\pgfpoint{\pgf@xa}{\pgf@ya}}
    \pgfpathlineto{\pgfpoint{\pgf@xb}{\pgf@ya}}
    % trinagle input
    \pgf@anchor@fdep@T
    \pgf@xa=\pgf@x \pgf@ya=\pgf@y
    \pgf@xb=\pgf@x \pgf@yb=\pgf@y
    \pgf@xc=\pgf@x \pgf@yc=\pgf@y
    \advance\pgf@ya by -.3333\pgf@x
    \advance\pgf@xb by -.5\pgf@x
    \advance\pgf@yb by .\pgf@x
    \advance\pgf@yc by .3333\pgf@x
    \pgfpathmoveto{\pgfpoint{\pgf@xa}{\pgf@ya}}
    \pgfpathlineto{\pgfpoint{\pgf@xb}{\pgf@yb}}
    \pgfpathlineto{\pgfpoint{\pgf@xc}{\pgf@yc}}
    \pgfclosepath

    % Draw port labels
    \begingroup
    \tikzset{fdep/port labels} % Use font from this style
    \tikz@textfont

    \pgf@anchor@fdep@O
    \pgftext[top,at={\pgfpoint{\pgf@x}{\pgf@y}},y=-\pgfshapeinnerysep]{}
    
    \pgf@anchor@fdep@EA
    \pgftext[bottom,at={\pgfpoint{\pgf@x}{\pgf@y}},y=\pgfshapeinnerysep]{}
    
    \pgf@anchor@fdep@EB
    \pgftext[bottom,at={\pgfpoint{\pgf@x}{\pgf@y}},y=\pgfshapeinnerysep]{}
    
    \pgf@anchor@fdep@EC
    \pgftext[bottom,at={\pgfpoint{\pgf@x}{\pgf@y}},y=\pgfshapeinnerysep]{}
    
    \pgf@anchor@fdep@ED
    \pgftext[bottom,at={\pgfpoint{\pgf@x}{\pgf@y}},y=\pgfshapeinnerysep]{}
    
    \pgf@anchor@fdep@EE
    \pgftext[bottom,at={\pgfpoint{\pgf@x}{\pgf@y}},y=\pgfshapeinnerysep]{}
    \endgroup
  }
}

\tikzset{add font/.code={\expandafter\def\expandafter\tikz@textfont\expandafter{\tikz@textfont#1}}} 

\tikzset{fdep/port labels/.style={font=\sffamily\scriptsize}}
\tikzset{every fdep node/.style={draw,minimum height=1cm,minimum 
width=1.5cm,inner sep=1mm,outer sep=0pt,cap=round,add 
font=\sffamily}}

\newif\ifpgfshapebaselesstrianglehasinline
\newif\ifpgfshapebaselesstriangleclose
\pgfkeys{/pgf/.cd,
  baseless triangle apex angle/.style={/pgf/isosceles triangle apex angle=#1},
  baseless triangle inline/.is if=pgfshapebaselesstrianglehasinline,
  baseless triangle has base/.is if=pgfshapebaselesstriangleclose
}

\pgfdeclareshape{baseless triangle}{
  % Copy some stuff from the isosecles triangle
  \inheritsavedanchors[from={isosceles triangle}]
  \inheritanchor[from={isosceles triangle}]{center}
  \inheritanchor[from={isosceles triangle}]{north}
  \inheritanchor[from={isosceles triangle}]{south}
  \inheritanchor[from={isosceles triangle}]{east}
  \inheritanchor[from={isosceles triangle}]{west}
  \inheritanchorborder[from={isosceles triangle}]
  \backgroundpath{%
    % The isoceles triangle defines lots of parameters
    % in the \trianglepoints macro.
        \trianglepoints%
        {%
            \pgftransformshift{\centerpoint}%
            \pgftransformrotate{\rotate}%
            % This bit is a bit of a kludge to ensure the inline
            % is at the top of the figure.
            \pgftransformyscale{cos(\rotate)}%
            \pgfpathmoveto{\lowerleft}%
            \pgfpathlineto{\apex}%
            \pgfpathlineto{\lowerleft\pgf@y=-\pgf@y}%
            % Close the base?
            \ifpgfshapebaselesstriangleclose%
              \pgfpathclose%
            \fi%
            % Draw the inline?
            \ifpgfshapebaselesstrianglehasinline
              \pgfpointdiff{\lowerleft}%
                 {\pgfpointlineattime{0.125}{\lowerleft}{\lowerleft\pgf@y=-\pgf@y}}%
                \pgfgetlastxy{\x}{\y}%
                \pgfmathveclen{\x}{\y}%
                \let\inlineshift=\pgfmathresult%
            % Calculate where the inline hits the sloped line of the triangle.
            \pgfmathparse{\inlineshift/2/sin(\pgfkeysvalueof{/pgf/isosceles triangle apex angle}/2)}%
            \let\inlineendshift=\pgfmathresult
            \pgfpathmoveto{\pgfpointadd{\pgfpoint{0pt}{-\inlineshift}}{\lowerleft}}%
            \pgfpathlineto{\pgfpointlineatdistance{\inlineendshift}{\apex}{\lowerleft}\pgf@y=-\pgf@y}%
        \fi%
    }
    }
}





\makeatother

\tikzstyle{every picture}=[>=stealth]
\tikzstyle{fdarrow}=[>=triangle 60,double,double equal sign distance,-implies,dashed]
\tikzstyle{and1}=[and gate US,draw,shape border rotate=90, rotate=90,logic gate inputs=n,minimum width=1cm]
\tikzstyle{and2}=[and gate US,draw,  shape border rotate=90, rotate=90,logic gate inputs=nn,minimum width=1cm]
\tikzstyle{and3}=[and gate US,draw, shape border rotate=90, rotate=90,logic gate inputs=nnn,minimum width=1cm]
\tikzstyle{and}=[and gate US,draw, shape border rotate=90, rotate=90,logic gate inputs=nnn,minimum width=1cm]
\tikzstyle{and4}=[and gate US,draw, shape border rotate=90, rotate=90,logic gate inputs=nnnn,minimum width=1cm, scale=0.9]
\tikzstyle{and5}=[and gate US,draw, shape border rotate=90, rotate=90,logic gate inputs=nnnnn,minimum width=1cm, scale=0.8]
\tikzstyle{and6}=[and gate US,draw, shape border rotate=90, rotate=90,logic gate inputs=nnnnnn,minimum width=1cm, scale=0.7]
\tikzstyle{or1}=[or gate US,draw,shape border rotate=90, rotate=90,logic gate inputs=n,minimum width=1cm]
\tikzstyle{or2}=[or gate US,draw,shape border rotate=90, rotate=90,logic gate inputs=nn,minimum width=1cm]
\tikzstyle{or3}=[or gate US,draw,shape border rotate=90, rotate=90,logic gate inputs=nnn,minimum width=1cm]
\tikzstyle{or}=[or gate US,draw,shape border rotate=90, rotate=90,logic gate inputs=nnn,minimum width=1cm]
\tikzstyle{or4}=[or gate US,draw,shape border rotate=90, rotate=90,logic gate inputs=nnnn,minimum width=1cm, scale=0.9]
\tikzstyle{or5}=[or gate US,draw,shape border rotate=90, rotate=90,logic gate inputs=nnnnn,minimum width=1cm, scale=0.8]
\tikzstyle{or6}=[or gate US,draw,shape border rotate=90, rotate=90,logic gate inputs=nnnnnn,minimum width=1cm, scale=0.8]
\tikzstyle{be}=[circle,draw,fill=white!100,anchor=north, inner sep=5pt, minimum width=0.7cm]
\tikzstyle{cnst}=[diamond,draw,fill=white!100,anchor=north, minimum width=0.5cm,font=\scriptsize, scale=0.8]
\tikzstyle{seq}=[rectangle, draw, fill=white!100, minimum width=1.0cm, minimum height=0.6cm, inner sep=3pt, label=center:{\LARGE$\rightarrow$}]
\tikzstyle{seqincl}=[seq, label={[shift={(0.35,-0.65)}]\small{$\leq$}}]
\tikzstyle{seqexcl}=[seq, label={[shift={(0.35,-0.65)}]\small{$<$}}]
\tikzstyle{mutex}=[rectangle, draw, fill=white!100, minimum width=1.0cm, minimum height=0.6cm, inner sep=3pt, label=center:{\LARGE$\leftrightarrow$}]
\tikzstyle{labelbox} = [rectangle, draw, inner sep=2pt, anchor=south, minimum width=0.6cm, minimum height=0.48cm, scale=0.8]
\tikzstyle{ratebox} = [rectangle, draw=none, inner sep=2pt, anchor=north, minimum width=0.6cm, minimum height=0.48cm, scale=0.8, align=center]
\tikzstyle{transfer} = [isosceles triangle, draw, shape border rotate=90, inner sep=2pt, minimum width=0.9cm, minimum height=0.7cm, isosceles triangle stretches=true]

\tikzstyle{triangle}=[isosceles triangle,draw,shape border rotate=90,isosceles triangle stretches=true, minimum height=0.6cm,minimum width=0.6cm]
\tikzstyle{triangle_pand}=[triangle, scale=1.62, yshift=-3.5, xscale=0.80]
\tikzstyle{btriangle}=[baseless triangle,draw,rotate=90, minimum height=0.6cm]
\tikzstyle{triangle_por}=[btriangle, scale=1.62, yscale=0.915, xshift=-0.113cm]

\tikzstyle{subtree}=[triangle]
\tikzstyle{failed}=[draw=red!60!black!100, fill=red!05]

\tikzstyle{repr}=[fill=black!30!white]
% Need to copy style from spare and explicitly set background colour
\tikzstyle{spare_repr}=[spare, every spare node/.style={draw,minimum height=1cm,minimum width=1.5cm,fill=black!30!white,inner sep=1mm,outer sep=0pt,cap=round,add font=\sffamily}]
